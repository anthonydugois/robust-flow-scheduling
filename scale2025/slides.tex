\documentclass[aspectratio=169]{beamer}

\usepackage{tikz}

\usetikzlibrary{positioning}

\DeclareMathOperator*{\argmin}{arg\,min}

\title{Preliminary Results on Robust Non-Clairvoyant\\Scheduling with Classification Models}
\author{Anthony Dugois\\\footnotesize (on-going work with Vincent Fagnon and Giorgio Lucarelli)}
\institute{Université Marie et Louis Pasteur, institut FEMTO-ST, Besançon}
\date{SCALE\\\footnotesize april 15, 2025}

\begin{document}


\frame{\titlepage}

\begin{frame}{Outline}
    \tableofcontents[hideallsubsections]
\end{frame}


\section{Model}

\begin{frame}{Outline}
    \tableofcontents[currentsection,hideallsubsections]
\end{frame}


\begin{frame}{Dealing with Uncertainty}

\begin{itemize}
    \item Stochastic models%; need assumptions on parameter distribution
    \item Min-max (regret) models%; make problems harder (sometimes not at all approximable)
    \item Learning-augmented models%; tied to error definition
\end{itemize}

\end{frame}


\begin{frame}{Classification Model}

\begin{itemize}
    \item Jobs are classified according to \(K\) classes
    \item Jobs of a given class \(k\) have similar characteristics (e.g., same processing times)
    \item The class of a job is unknown
    \item Access to an oracle based on a classification model, which \textbf{can make mistakes}
\end{itemize}

\end{frame}


\begin{frame}{Confusion Matrix}

\begin{columns}
    \begin{column}{.5\textwidth}
        The oracle is represented as a \(K\times K\) \emph{confusion matrix}:
        \[
            E=\begin{pmatrix}
                1 & 1 & 3 \\
                2 & 2 & 1 \\
                1 & 1 & 2 \\
            \end{pmatrix}
        \]
    \end{column}
    \begin{column}{.5\textwidth}
        \begin{itemize}
            \item Row \(i\): jobs that the oracle \emph{believes} to be in class \(i\)
            \item Column \(j\): jobs \emph{actually} in class \(j\)
            \item Entry \(e_{ij}\): number of jobs believed to be in \(i\) but actually in \(j\)
            \item We suppose that \(E\) is known
        \end{itemize}
    \end{column}
\end{columns}

\end{frame}


\begin{frame}{A First Scheduling Problem}

\begin{itemize}
    \item \(1||\sum C_j\) with \(K\) processing times \(p(1)\le\cdots\le p(K)\)
    \item SPT is optimal when all processing times are known
    \item In a non-clairvoyant setting, the only solution is to schedule jobs randomly
\end{itemize}

\vspace{20pt}

\begin{block}{Question}
    What can we do when we have access to an oracle (and its confusion matrix)?
\end{block}

\end{frame}


\begin{frame}{A First Scheduling Problem}

\begin{columns}
    \begin{column}{.4\textwidth}
        \centering

        \[
            E=\begin{pmatrix}
                \textcolor{green!70!black}{1} & \textcolor{orange!70!black}{1} & \textcolor{red!70!black}{3} \\
                \textcolor{green!70!black}{2} & \textcolor{orange!70!black}{2} & \textcolor{red!70!black}{1} \\
                \textcolor{green!70!black}{1} & \textcolor{orange!70!black}{1} & \textcolor{red!70!black}{2} \\
            \end{pmatrix}
        \]

        \(\textcolor{green!70!black}{p(1)=1},\textcolor{orange!70!black}{p(2)=4},\textcolor{red!70!black}{p(3)=6}\)
    \end{column}
    \begin{column}{.3\textwidth}
        \centering
        Seen by oracle

        \vspace{20pt}

        \begin{tikzpicture}[
            default/.style={
                inner sep=0,
                outer sep=0,
                anchor=south,
                draw=white
            }
        ]
            \def\x{12pt}
            \def\y{6pt}

            \begin{scope}
                \node[
                    default,
                    minimum width=\x,
                    minimum height=6*\y
                ] at (0,0) {};
                \node[
                    default,
                    minimum width=\x,
                    minimum height=\y,
                    fill=green!70!black
                ] at (0,0) {};

                \node[
                    default,
                    minimum width=\x,
                    minimum height=\y,
                    fill=green!70!black
                ] at (\x,0) {};

                \node[
                    default,
                    minimum width=\x,
                    minimum height=\y,
                    fill=green!70!black
                ] at (2*\x,0) {};

                \node[
                    default,
                    minimum width=\x,
                    minimum height=\y,
                    fill=green!70!black
                ] at (3*\x,0) {};

                \node[
                    default,
                    minimum width=\x,
                    minimum height=\y,
                    fill=green!70!black
                ] at (4*\x,0) {};
            \end{scope}

            \begin{scope}[yshift=-8*\y]
                \node[
                    default,
                    minimum width=\x,
                    minimum height=3*\y,
                    fill=orange!70!black
                ] at (0,0) {};

                \node[
                    default,
                    minimum width=\x,
                    minimum height=3*\y,
                    fill=orange!70!black
                ] at (\x,0) {};

                \node[
                    default,
                    minimum width=\x,
                    minimum height=3*\y,
                    fill=orange!70!black
                ] at (2*\x,0) {};

                \node[
                    default,
                    minimum width=\x,
                    minimum height=3*\y,
                    fill=orange!70!black
                ] at (3*\x,0) {};

                \node[
                    default,
                    minimum width=\x,
                    minimum height=3*\y,
                    fill=orange!70!black
                ] at (4*\x,0) {};
            \end{scope}

            \begin{scope}[yshift=-16*\y]
                \node[
                    default,
                    minimum width=\x,
                    minimum height=6*\y,
                    fill=red!70!black
                ] at (0,0) {};

                \node[
                    default,
                    minimum width=\x,
                    minimum height=6*\y,
                    fill=red!70!black
                ] at (\x,0) {};

                \node[
                    default,
                    minimum width=\x,
                    minimum height=6*\y,
                    fill=red!70!black
                ] at (2*\x,0) {};

                \node[
                    default,
                    minimum width=\x,
                    minimum height=6*\y,
                    fill=red!70!black
                ] at (3*\x,0) {};
            \end{scope}
        \end{tikzpicture}
    \end{column}
    \begin{column}{.3\textwidth}
        \centering
        Reality

        \vspace{20pt}

        \begin{tikzpicture}[
            default/.style={
                inner sep=0,
                outer sep=0,
                anchor=south,
                draw=white
            }
        ]
            \def\x{12pt}
            \def\y{6pt}

            \begin{scope}
                \node[
                    default,
                    minimum width=\x,
                    minimum height=6*\y,
                    fill=red!70!black
                ] at (0,0) {};

                \node[
                    default,
                    minimum width=\x,
                    minimum height=4*\y,
                    fill=orange!70!black
                ] at (\x,0) {};

                \node[
                    default,
                    minimum width=\x,
                    minimum height=\y,
                    fill=green!70!black
                ] at (2*\x,0) {};

                \node[
                    default,
                    minimum width=\x,
                    minimum height=6*\y,
                    fill=red!70!black
                ] at (3*\x,0) {};

                \node[
                    default,
                    minimum width=\x,
                    minimum height=6*\y,
                    fill=red!70!black
                ] at (4*\x,0) {};
            \end{scope}

            \begin{scope}[yshift=-8*\y]
                \node[
                    default,
                    minimum width=\x,
                    minimum height=6*\y,
                    fill=red!70!black
                ] at (0,0) {};

                \node[
                    default,
                    minimum width=\x,
                    minimum height=3*\y,
                    fill=orange!70!black
                ] at (\x,0) {};

                \node[
                    default,
                    minimum width=\x,
                    minimum height=\y,
                    fill=green!70!black
                ] at (2*\x,0) {};

                \node[
                    default,
                    minimum width=\x,
                    minimum height=3*\y,
                    fill=orange!70!black
                ] at (3*\x,0) {};

                \node[
                    default,
                    minimum width=\x,
                    minimum height=\y,
                    fill=green!70!black
                ] at (4*\x,0) {};
            \end{scope}

            \begin{scope}[yshift=-16*\y]
                \node[
                    default,
                    minimum width=\x,
                    minimum height=6*\y,
                    fill=red!70!black
                ] at (0,0) {};

                \node[
                    default,
                    minimum width=\x,
                    minimum height=3*\y,
                    fill=orange!70!black
                ] at (\x,0) {};

                \node[
                    default,
                    minimum width=\x,
                    minimum height=\y,
                    fill=green!70!black
                ] at (2*\x,0) {};

                \node[
                    default,
                    minimum width=\x,
                    minimum height=6*\y,
                    fill=red!70!black
                ] at (3*\x,0) {};
            \end{scope}
        \end{tikzpicture}
    \end{column}
\end{columns}

\end{frame}


\begin{frame}{A First Scheduling Problem}

\begin{itemize}
    \item The oracle gives us rows \(B(1),\ldots,B(K)\), where \(B(k)\) contains the jobs that the oracle believes to be in class \(k\)
    \item We know the distribution in each \(B(k)\) according to the matrix \(E\), but we cannot actually distinguish jobs
    \item The feasible strategies all consist in choosing a sequence of rows from which randomly picking a job
    % \item The feasible strategies all consist in choosing a sequence \(r=(r(1),r(2),\ldots,r(n))\), where \(r(t)\in\{1,\ldots,K\}\) denotes the set from which we pick a random job at step \(t\)
\end{itemize}

\vspace{20pt}

\begin{block}{Question}
    How to choose an optimal sequence of rows?
\end{block}

\end{frame}


\begin{frame}{Refined Problem}
    \begin{itemize}
        \item Picking a random job in a set can be seen as considering a fixed ordering of the jobs and pulling the job located at the head
        \item Consider an (unknown) scenario \(\sigma\), which fixes the ordering of the jobs in each row (let \(\sigma_k\) denote the considered permutation of \(B(k)\))
        \item The goal is to choose a sequence \(r=(r(1),r(2),\ldots,r(n))\), where \(r(t)\in\{1,\ldots,K\}\) denotes the row from which we pull a job at step \(t\), that minimizes \(\sum C_j\)
    \end{itemize}
\end{frame}


\section{Some Results}

\begin{frame}{Outline}
    \tableofcontents[currentsection,hideallsubsections]
\end{frame}


\begin{frame}{Adaptive vs.\ Non-Adaptive Strategies}

\begin{itemize}
    \item Adaptive: the strategy adapts the sequence \(r\) as it learns about the already-executed jobs (i.e., it learns about \(\sigma\))
    \item \textbf{Non-adaptive}: the strategy must decide the sequence \(r\) before executing any job, without knowing \(\sigma\)
\end{itemize}

\end{frame}


\begin{frame}{Objectives}

Let \(R\) be the set of feasible sequences, \(S\) the set of scenarios, and \(\mathcal{C}(r,\sigma)\) the objective of the schedule generated from sequence \(r\) when applied on scenario \(\sigma\).

\vspace{20pt}

\begin{itemize}
    \item Min-Max: \(\min_{r\in R} \max_{\sigma\in S} \mathcal{C}(r,\sigma)\)
    \item Min-Average: \(\min_{r\in R} \sum_{\sigma\in S} \mathcal{C}(r,\sigma)\)
    \item Min-Max Regret: \(\min_{r\in R} \max_{\sigma\in S} \left(\mathcal{C}(r,\sigma)-\mathcal{C}^*(\sigma)\right)\), where \(\mathcal{C}^*(\sigma)\) is the optimal solution for \(\sigma\)
\end{itemize}

\end{frame}


\begin{frame}{Optimal Sequence for a Fixed \(\sigma\)}

\begin{itemize}
    \item For a fixed scenario \(\sigma\), the problem is equivalent to \(1|\text{chains}|\sum C_j\)
    \item The following algorithm is optimal:
    \begin{enumerate}
        \item Compute \(x(k,j)=\frac{1}{j} \sum_{i=1}^j p_{\sigma_k(i)}\) for all \(1\le k\le K\) and all \(1\le j\le |B(k)|\), where \(p_{\sigma_k(i)}\) is the processing time of the \(i\)-th job in \(B(k)\)
        \item Compute \(y(k)=\min_j x(k,j)\), \(h(k)=\argmin_j x(k,j)\) for all \(k\)
        \item Let \(k^*=\argmin_k y(k)\) and schedule the first \(h(k^*)\) jobs of \(B(k^*)\)
        \item Repeat from step 1
    \end{enumerate}
\end{itemize}

\end{frame}


\begin{frame}{Min-Max: \(\min_{r\in R} \max_{\sigma\in S} \mathcal{C}(r,\sigma)\)}

\begin{lemma}
    For any fixed sequence \(r\), the worst scenario consists in jobs of each row \(B(k)\) being arranged in decreasing order of processing times.
\end{lemma}

\begin{theorem}
    Let \(\bar{p}(k)=\sum_{j=1}^K \frac{e_{kj}}{|B(k)|}p(j)\) be the average processing time of jobs in \(B(k)\).
    Scheduling sets \(B(k)\) in increasing order of \(\bar{p}(k)\) solves Min-Max.
\end{theorem}

\end{frame}


\begin{frame}{Min-Average: \(\min_{r\in R} \sum_{\sigma\in S} \mathcal{C}(r,\sigma)\)}

\begin{theorem}
    Scheduling sets \(B(k)\) in increasing order of \(\bar{p}(k)\) solves Min-Average.
\end{theorem}

\vspace{20pt}

\begin{block}{Open questions}
    \begin{itemize}
        \item Generalized problem \(\min_{r\in R} \left(\sum_{\sigma\in S} \mathcal{C}(r,\sigma)^p\right)^{1/p}\)
        \item Generalized problem \(\min_{r\in R} \sum_{\sigma\in S} w_\sigma \mathcal{C}(r,\sigma)\)
    \end{itemize}
\end{block}

\end{frame}


\begin{frame}{Min-Max Regret: \(\min_{r\in R} \max_{\sigma\in S} \left(\mathcal{C}(r,\sigma)-\mathcal{C}^*(\sigma)\right)\)}

\begin{block}{Open questions}
    \begin{itemize}
        \item Is the problem NP-hard?
        \item For a fixed sequence \(r\), can we find the scenario \(\sigma\) maximizing the regret \(\mathcal{C}(r,\sigma)-\mathcal{C}^*(\sigma)\)?
    \end{itemize}
\end{block}

\end{frame}


\section{On-Going Work}

\begin{frame}{Outline}
    \tableofcontents[currentsection,hideallsubsections]
\end{frame}


\begin{frame}{On-Going (and Future) Work}

\begin{itemize}
    \item Min-Max Regret
    \item Adaptive strategies
    \item More complex scheduling problems, e.g., \(P||\sum C_j\), \(1|r_j|\sum C_j\), \(1||\sum w_jC_j\)
    \item Are there easier oracles than others?
\end{itemize}

\end{frame}


\end{document}
