\documentclass[11pt]{article}

%\usepackage{carlito}

\usepackage[tmargin=3.5cm,bmargin=2cm,inner=2.1cm,outer=2.1cm,headheight=1.5cm]{geometry}

% TODO notes
\usepackage{todonotes}
\newcommand{\AD}[2][inline]{\todo[caption={},color=blue!50,#1]{\small\sf\textbf{FP:} #2}}
\newcommand{\VF}[2][inline]{\todo[caption={},color=green!30,#1]{\small\sf\textbf{EB:} #2}}
\newcommand{\GL}[2][inline]{\todo[caption={},color=red!40,#1]{\small\sf\textbf{GL:} #2}}

\usepackage{fancyhdr}
\renewcommand{\headruleskip}{1mm}

\renewcommand{\thesection}{\Roman{section}.}

\begin{document}

\pagestyle{fancy}

\fancyhead[L]{
	\begin{tabular}{l}
		\textbf{AAPG2026} \\
		Coordinated by: \\
		Number and title of the chosen scientific theme (to be found in 2026 AAPG call)
	\end{tabular}
}

\fancyhead[C]{
	\begin{tabular}{l}
		\textbf{ACRONYM} \\
		First name SURNAME \\
		~
	\end{tabular}
}

\fancyhead[R]{
	\begin{tabular}{l}
		Instrument \\
		Duration \\
		~
	\end{tabular}
}


\centerline{\Large \textbf{Title of the project}}

\section{Context, positioning and objective(s) of the project}

Assuming that we have a complete knowledge of the input instance is not a valid hypothesis in most practical situations, despite the theoretical and structural interest of studying such a case.
On the other hand, considering full absence of knowledge is quite pessimistic in the era of big data analysis and machine learning.
In the direction of shortening this gap, several models have been proposed, such as using not always correct machine learned predictions~\cite{LykourisV18}, querying an oracle at an extra cost~\cite{DurrEMM18,Kahan91}, stochastic approaches~\cite{}, etc.
In this project, inspired by the principle of Bloom's filter~\cite{Bloom70}, we propose to focus on an unreliable oracle whose predictions errors respect some structural assumptions, i.e., absence of false negatives.
Our objective during this project is to apply this kind of oracle on scheduling problems.
This is motivated by the importance and the wide range of applications of scheduling on both operational research and parallel computing domains.

\GL{Next : A paragraph explaining Bloom's filter and how to extend and adapt it to our context}

\paragraph{Related work}

\paragraph{Methodology}

\GL{1. current work (regret, expectation, etc)\\
2. competitive ratio as a function of prediction error}

\section{Partnership (consortium or team)}

%\section{References}

\renewcommand\refname{III.\quad References}
\bibliographystyle{plain}
\bibliography{biblio.bib}

\end{document}