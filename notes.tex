\documentclass{article}

\usepackage[utf8]{inputenc}
\usepackage[T1]{fontenc}
\usepackage{etoolbox}
\usepackage{amsthm}
\usepackage{amsmath}
\usepackage{amssymb}
\usepackage{algorithm}
\usepackage[noend]{algpseudocode}
\usepackage{booktabs}

\usepackage{hyperref}
\hypersetup{
    citecolor=blue,
    urlcolor=blue,
    linkcolor=blue,
    colorlinks=true
}

%%% Environment for comments.
%%% Set the boolean to false to produce a comment-free version.
\newtoggle{showcomments}
\settoggle{showcomments}{true}
\iftoggle{showcomments}{
    \newcommand{\definecomment}[3]{%
        \fbox{\bfseries\sffamily\scriptsize #1}%
        ~{\small\textsf{\emph{\color{#3}{#2}}}}}
}{
    \newcommand{\definecomment}[3]{}
}

%%% One command per author.
\newcommand{\ad}[1]{\definecomment{AD}{#1}{red}}
\newcommand{\gl}[1]{\definecomment{GL}{#1}{blue}}

\title{Notes on Robustness in Flow Time Scheduling for Two Classes of Jobs}

\begin{document}

\maketitle

\section{Model}

\subsection{Problem}

\begin{itemize}
    \item \(n\) jobs
    \item \(m\) identical machines
    \item release times \(r_j\)
    \item processing times \(p_j\in\{1,L\}\) (small/large)
    \item no preemption
    \item non-clairvoyant
    \item objective \(\sum F_j\)
\end{itemize}

\subsection{Oracle}

We note \(n_S\) (resp.\ \(n_L\)) the number of small (resp.\ large) jobs. The oracle encodes the set
of large jobs, i.e., we can query the oracle to know if a job is large or not. There is no cost for
asking the oracle. However, the oracle is not always correct. We consider asymmetrical errors, that
is, there is a probability \(p\) that the oracle is wrong when a job is small, and a probability
\(q\) that the oracle is wrong when a job is large. The oracle is deterministic, i.e., it always
gives the same answer for a given job. Put differently, there are \(\tilde{n}_S\) small jobs that
the oracle believes to be large, and \(\tilde{n}_L\) large jobs that the oracle believes to be
small.

\section{Related Work}

\begin{itemize}
    \item \(R||\sum C_j\) is in P \cite{bruno1974scheduling}
    \item \(1|r_j|\sum C_j\) is strongly NP-hard \cite{lenstra1977complexity}
    \item \(1|r_j,\mathit{pmtn}|\sum F_j\) is in P (equivalent to \(1|r_j,\mathit{pmtn}|\sum C_j\),
    solved by SRPT)
    \item \(P|r_j,p_j=p|\sum C_j\) is in P \cite{simons1983multiprocessor}
    \item \(P|r_j,\mathit{pmtn}|\sum C_j\) is strongly NP-hard \cite{bellenguez2015note}
    \item FCFS is \(\Delta\)-competitive (\(\Delta=\max p_j/\min p_j\)) for \(1|r_j|\sum F_j\)
    \cite{legrand2008minimizing}
    \item There is a \(\frac{e}{e-1}\)-approximation algorithm for \(1|r_j|\sum C_j\) \cite{chekuri2001approximation}
    \item No deterministic online algorithm has a competitive ratio lower than 2 for \(1|r_j|\sum
    C_j\) \cite{chekuri2001approximation}
    \item There is a \(\frac{e}{e-1}\)-competitive randomized algorithm for \(1|r_j|\sum C_j\)
    \cite{chekuri2001approximation}
    \item There is a 2-approximation algorithm for \(P|r_j|\sum w_jC_j\) and
    \(P|r_j,\mathit{pmtn}|\sum w_jC_j\) \cite{schulz1997scheduling}
    \item There is a \(O(\sqrt{n})\)-approximation algorithm for \(1|r_j|\sum F_j\)
    \cite{kellerer1996approximability}
    \item For any \(\varepsilon>0\), there is no \(O(n^{\frac{1}{2}-\varepsilon})\)-approximation
    algorithm for \(1|r_j|\sum F_j\) unless P=NP \cite{kellerer1996approximability}
    \item SRPT is an \(O(\log\min(n/m,\Delta))\)-approximation algorithm for
    \(P|r_j,\mathit{pmtn}|\sum F_j\) \cite{leonardi1997approximating}
    \item No randomized online algorithm has a competitive ratio lower than
    \(\Omega(\log(n/m+\Delta))\) for \(P|r_j,\mathit{pmtn}|\sum F_j\)
    \cite{leonardi1997approximating}
    \item For any \(\varepsilon>0\), there is no \(O(n^{\frac{1}{3}-\varepsilon})\)-approximation
    algorithm for \(P|r_j|\sum F_j\) unless P=NP \cite{leonardi1997approximating}
\end{itemize}

\subsection{Non-clairvoyant}

\begin{itemize}
    \item In a non-clairvoyant setting, RR is a \((2-\frac{2}{n+1})\)-competitive algorithm for
    \(1|\mathit{pmtn}|\sum C_j\) and a \((2-\frac{2m}{(n+m)})\)-competitive algorithm for
    \(P|\mathit{pmtn}|\sum C_j\) \cite{motwani1994nonclairvoyant}
    \item No deterministic non-clairvoyant algorithm has a competitive ratio lower than
    \(2-\frac{2}{n+1}\) for \(1|\mathit{pmtn}|\sum C_j\) and \(2-\frac{2m}{(n+m)}\) for
    \(P|\mathit{pmtn}|\sum C_j\) \cite{motwani1994nonclairvoyant}
    \item There is a \((2-\frac{4}{n+3})\)-competitive randomized algorithm for
    \(1|\mathit{pmtn}|\sum C_j\) in a non-clairvoyant setting \cite{motwani1994nonclairvoyant}
    \item No randomized non-clairvoyant algorithm has a competitive ratio lower than
    \(2-\frac{4}{n+3}\) for \(1|\mathit{pmtn}|\sum C_j\) \cite{motwani1994nonclairvoyant}
    \item No deterministic non-clairvoyant algorithm has a competitive ratio lower than
    \(\Omega(n^{\frac{1}{3}})\) for \(1|r_j,\mathit{pmtn}|\sum F_j\)
    \cite{motwani1994nonclairvoyant}
    \item No randomized non-clairvoyant algorithm has a competitive ratio lower than \(\Omega(\log
    n)\) for \(1|r_j,\mathit{pmtn}|\sum F_j\)
    \cite{motwani1994nonclairvoyant}
    \item RMLF is a \(O(\log n\log\log n)\)-competitive algorithm for \(1|r_j,\mathit{pmtn}|\sum
    F_j\) in the non-clairvoyant setting \cite{kalyanasundaram2003minimizing}
    \item RMLF is a \(O(\log n\log\min(n/m,\Delta))\)-competitive algorithm for
    \(P|r_j,\mathit{pmtn}|\sum F_j\) in the non-clairvoyant setting
    \cite{becchetti2004nonclairvoyant}
\end{itemize}

\subsection{Learning-augmented}

\begin{itemize}
    \item There is a \(O(1/\varepsilon)\)-robust and \((1+\varepsilon)\)-consistent algorithm (the
    expected objective value is at most
    \(\min\{O(1/\varepsilon)\mathrm{OPT},(1+\varepsilon)\mathrm{OPT}+O(1/\varepsilon^3\log(1/\varepsilon))v\}\),
    where \(v\) is the prediction error of the oracle), for \(1|\mathit{pmtn}|\sum C_j\)
    \cite{im2023non}
    \item There is a \(3\Delta\)-robust and 1-consistent algorithm for \(1|r_j,\mathit{pmtn}|\sum
    F_j\) \cite{zhao2022brief}
\end{itemize}

\section{Sum of Completion Times, No Release Times}

\subsection{Single Machine \(1||\sum C_j\)}

It is well-known that in the clairvoyant case, the optimal solution consists in scheduling small
jobs first, i.e., the optimal objective value is
\[
    \sum C_j^*=\frac{1}{2}n_S(n_S+1)+n_Sn_L+\frac{1}{2}n_L(n_L+1)L.
\]

In the non-clairvoyant case (NC), we clearly cannot apply the same strategy. By taking a random
permutation (uniformly) on the jobs, we can show that the expected value of the objective is
\[
    E(\sum C_j^{\mathrm{NC}})=\frac{1}{2}(n_S+n_L+1)(n_S+n_LL).
\]

Now let us consider our semi-clairvoyant oracle (SC). If we trust it and try to apply the optimal
clairvoyant strategy, we can easily derive the objective value of a worst-case schedule, which
consists in scheduling the \(\tilde{n}_L\) large jobs that the oracle believes to be small first,
then the \(n_S-\tilde{n}_S\) small jobs, then the \(n_L-\tilde{n}_L\) large jobs, and finally the
\(\tilde{n}_S\) small jobs that the oracle believes to be large. Then, for any instance, we have
\begin{align*}
    \sum C_j^{\mathrm{SC}}
        &\le\frac{1}{2}\tilde{n}_L(\tilde{n}_L+1)L\\
        &\qquad+(n_S-\tilde{n}_S)\tilde{n}_LL+\frac{1}{2}(n_S-\tilde{n}_S)(n_S-\tilde{n}_S+1)\\
        &\qquad+(n_L-\tilde{n}_L)(n_S-\tilde{n}_S+\tilde{n}_LL)+\frac{1}{2}(n_L-\tilde{n}_L)(n_L-\tilde{n}_L+1)L\\
        &\qquad+\tilde{n}_S(n_S-\tilde{n}_S+n_LL)+\frac{1}{2}\tilde{n}_S(\tilde{n}_S+1).
\end{align*}
This bound constitutes the general guarantee permitted by the oracle. By shuffling independently the
\(n_S-\tilde{n}_S+\tilde{n}_L\) small jobs (as seen by the oracle) and the
\(\tilde{n}_S+n_L-\tilde{n}_L\) large jobs (as seen by the oracle), we get
\begin{align*}
    E(\sum C_j^{\mathrm{SC}})
        &=\frac{1}{2}(n_S-\tilde{n}_S+\tilde{n}_L+1)(n_S-\tilde{n}_S+\tilde{n}_LL)\\
        &\qquad+(\tilde{n}_S+n_L-\tilde{n}_L)(n_S-\tilde{n}_S+\tilde{n}_LL)\\
        &\qquad+\frac{1}{2}(\tilde{n}_S+n_L-\tilde{n}_L+1)(\tilde{n}_S+n_LL-\tilde{n}_LL)
\end{align*}

However, it is clearly not robust when the oracle is of poor quality. This can be easily seen by
considering the most extreme situation where the oracle is wrong for all jobs, i.e.,
\(\tilde{n}_S=n_S\) and \(\tilde{n}_L=n_L\). In this case, large jobs are scheduled first, and it is
better to simply apply the non-clairvoyant shuffling strategy.

% Instead, consider the following randomized algorithm. In the first step, we treat the
% \(n_S-\tilde{n}_S+\tilde{n}_L\) small jobs (as seen by the oracle). Let \(\sigma_S=n_S-\tilde{n}_S\)
% (resp.\ \(\sigma_L=\tilde{n}_L\)) be the number of jobs that are actually small (resp.\ large) among
% them. Then, schedule each job \(j\) as follows:
% \begin{itemize}
%     \item execute \(j\) immediately with probability \(\frac{\sigma_S}{\sigma_S+\sigma_L}\), and set
%     \(\sigma_S\) to \(\sigma_S-1\);
%     \item postpone \(j\) with probability \(\frac{\sigma_L}{\sigma_S+\sigma_L}\), and set
%     \(\sigma_L\) to \(\sigma_L-1\).
% \end{itemize}
% Repeat the process in the second step, starting with \(\sigma_S=\tilde{n}_S\) and
% \(\sigma_L=n_L-\tilde{n}_L\). Finally, execute all postponed jobs at the end.

To derive a better strategy, we can use the following observation. Among the
\(n_S-\tilde{n}_S+\tilde{n}_L\) small jobs (as seen by the oracle), we know that there are
\(n_S-\tilde{n}_S\) actual small jobs that must be immediately executed, and \(\tilde{n}_L\) actual
large jobs that must be postponed. Similarly, among the \(\tilde{n}_S+n_L-\tilde{n}_L\) large jobs
(as seen by the oracle), we know that there are \(\tilde{n}_S\) actual small jobs that must be
executed immediately, and \(n_L-\tilde{n}_L\) actual large jobs that must be postponed.

Thus, in the first step (the small jobs given by the oracle), pick \(n_S-\tilde{n}_S\) jobs among
\(n_S-\tilde{n}_S+\tilde{n}_L\) at random (uniformly), and execute them immediately. Then, postpone
the remaining jobs at the end. Perform the same process in the second step, by picking
\(\tilde{n}_S\) jobs among \(\tilde{n}_S+n_L-\tilde{n}_L\) to execute them, and postponing the rest.

For a given schedule, we can write
\begin{align*}
    \sum C_j
        &=(n_S-\tilde{n}_S)p_1+(n_S-\tilde{n}_S-1)p_2+\ldots+p_{n_S-\tilde{n}_S}\\
        &\qquad+\tilde{n}_S(\sum\limits_{j=1}^{n_S-\tilde{n}_S} p_j)\\
        &\qquad+\tilde{n}_Sp_{n_S-\tilde{n}_S+1}+(\tilde{n}_S-1)p_{n_S-\tilde{n}_S+2}+\ldots+p_{n_S}\\
        &\qquad+\tilde{n}_L(\sum\limits_{j=1}^{n_S} p_j)\\
        &\qquad+\tilde{n}_Lp_{n_S+1}+(\tilde{n}_L-1)p_{n_S+2}+\ldots+p_{n_S+\tilde{n}_L}\\
        &\qquad+(n_L-\tilde{n}_L)(\sum\limits_{j=1}^{n_S+\tilde{n}_L} p_j)\\
        &\qquad+(n_L-\tilde{n}_L)p_{n_S+\tilde{n}_L+1}+(n_L-\tilde{n}_L-1)p_{n_S+\tilde{n}_L+2}+\ldots+p_{n_S+n_L}.
\end{align*}
The first line is the sum of completion times for the \(n_S-\tilde{n}_S\) jobs executed in the first
step. The second and third lines are the sum of completion times for the \(\tilde{n}_S\) jobs
executed in the second step. The fourth and fifth lines are the sum of completion times for the
\(\tilde{n}_L\) jobs postponed in the first step. Finally, the sixth and seventh lines are the sum
of completion times for the \(n_L-\tilde{n}_L\) jobs postponed in the second step.

In the first step, we have
\(E(p_j)=\frac{n_S-\tilde{n}_S+\tilde{n}_LL}{n_S-\tilde{n}_S+\tilde{n}_L}\). In the second step, we
have \(E(p_j)=\frac{\tilde{n}_S+(n_L-\tilde{n}_L)L}{\tilde{n}_S+n_L-\tilde{n}_L}\). Therefore,

\begin{align*}
    E(\sum C_j)
        &=\frac{(n_S-\tilde{n}_S)(n_S-\tilde{n}_S+1)(n_S-\tilde{n}_S+\tilde{n}_LL)}{2(n_S-\tilde{n}_S+\tilde{n}_L)}\\
        &\qquad+\tilde{n}_S(n_S-\tilde{n}_S)\frac{n_S-\tilde{n}_S+\tilde{n}_LL}{n_S-\tilde{n}_S+\tilde{n}_L}\\
        &\qquad+\frac{\tilde{n}_S(\tilde{n}_S+1)(\tilde{n}_S+(n_L-\tilde{n}_L)L)}{2(\tilde{n}_S+n_L-\tilde{n}_L)}\\
        &\qquad+\tilde{n}_L((n_S-\tilde{n}_S)\frac{n_S-\tilde{n}_S+\tilde{n}_LL}{n_S-\tilde{n}_S+\tilde{n}_L}+\tilde{n}_S\frac{\tilde{n}_S+(n_L-\tilde{n}_L)L}{\tilde{n}_S+n_L-\tilde{n}_L})\\
        &\qquad+\frac{\tilde{n}_L(\tilde{n}_L+1)(n_S-\tilde{n}_S+\tilde{n}_LL)}{2(n_S-\tilde{n}_S+\tilde{n}_L)}\\
        &\qquad+(n_L-\tilde{n}_L)((n_S-\tilde{n}_S)\frac{n_S-\tilde{n}_S+\tilde{n}_LL}{n_S-\tilde{n}_S+\tilde{n}_L}+\tilde{n}_S\frac{\tilde{n}_S+(n_L-\tilde{n}_L)L}{\tilde{n}_S+n_L-\tilde{n}_L}+\tilde{n}_L\frac{n_S-\tilde{n}_S+\tilde{n}_LL}{n_S-\tilde{n}_S+\tilde{n}_L})\\
        &\qquad+\frac{(n_L-\tilde{n}_L)(n_L-\tilde{n}_L+1)(\tilde{n}_S+(n_L-\tilde{n}_L)L)}{2(\tilde{n}_S+n_L-\tilde{n}_L)}.
\end{align*}

Now we notice that, when we execute a set of job, we learn about the number of jobs that are
actually small. We can use this information to improve the execution of the next set of jobs.

\begin{algorithm}
    \begin{algorithmic}[1]
        \State \(P_1\gets\) the set of \(n_S-\tilde{n}_S+\tilde{n}_L\) small jobs (as seen by the oracle)
        \State \(P_2\gets\) the set of \(\tilde{n}_S+n_L-\tilde{n}_L\) large jobs (as seen by the oracle)
        \State \(\sigma_S\gets n_S-\tilde{n}_S\)
        \State \(\tilde{\sigma}_S\gets \tilde{n}_S\)
        \While{\(\sigma_S>0\) or \(\tilde{\sigma}_S>0\)}
            \If{\(\sigma_S>0\)}
                \State \(E_1\gets\) pick \(\sigma_S\) jobs among \(P_1\) at random (uniformly)
                \State Execute \(E_1\)
                \State Remove the number of small jobs seen in \(E_1\) from \(\sigma_S\)
                \State \(P_1\gets P_1\setminus E_1\)
            \EndIf
            \If{\(\tilde{\sigma}_S>0\)}
                \State \(E_2\gets\) pick \(\tilde{\sigma}_S\) jobs among \(P_2\) at random (uniformly)
                \State Execute \(E_2\)
                \State Remove the number of small jobs seen in \(E_2\) from \(\tilde{\sigma}_S\)
                \State \(P_2\gets P_2\setminus E_2\)
            \EndIf
        \EndWhile
        \State Execute \(P_1\) and \(P_2\)
    \end{algorithmic}
\end{algorithm}

\subsection{Identical Machines \(P||\sum C_j\)}

\ad{TODO}

\section{Sum of Completion Times, Release Times}

\ad{TODO}

\section{Sum of Flow Times, Release Times}

\ad{TODO}

\nocite{*}
\bibliographystyle{plain}
\bibliography{refs}

\end{document}
