\documentclass{article}

\usepackage[utf8]{inputenc}
\usepackage[T1]{fontenc}
\usepackage{etoolbox}
\usepackage{amsthm}
\usepackage{amsmath}
\usepackage{amssymb}
\usepackage{algorithm}
\usepackage[noend]{algpseudocode}
\usepackage{booktabs}

\usepackage{hyperref}
\hypersetup{
    citecolor=blue,
    urlcolor=blue,
    linkcolor=blue,
    colorlinks=true
}

%%% Environment for comments.
%%% Set the boolean to false to produce a comment-free version.
\newtoggle{showcomments}
\settoggle{showcomments}{true}
\iftoggle{showcomments}{
    \newcommand{\definecomment}[3]{%
        \fbox{\bfseries\sffamily\scriptsize #1}%
        ~{\small\textsf{\emph{\color{#3}{#2}}}}}
}{
    \newcommand{\definecomment}[3]{}
}

%%% One command per author.
\newcommand{\ad}[1]{\definecomment{AD}{#1}{red}}
\newcommand{\gl}[1]{\definecomment{GL}{#1}{blue}}

\title{Notes on Robustness in Flow Time Scheduling for Two Classes of Jobs}

\begin{document}

\maketitle

\section{Model}

\subsection{Problem}

\begin{itemize}
    \item \(n\) jobs
    \item \(m\) identical machines
    \item release times \(r_j\)
    \item processing times \(p_j\in\{1,L\}\) (small/large)
    \item no preemption
    \item non-clairvoyant
    \item objective \(\sum F_j\)
\end{itemize}

\subsection{Oracle}

We note \(n_S\) (resp.\ \(n_L\)) the number of small (resp.\ large) jobs. The oracle encodes the set
of large jobs, i.e., we can query the oracle to know if a job is large or not. There is no cost for
asking the oracle. However, the oracle is not always correct. We consider asymmetrical errors, that
is, there is a probability \(p\) that the oracle is wrong when a job is small, and a probability
\(q\) that the oracle is wrong when a job is large. The oracle is deterministic, i.e., it always
gives the same answer for a given job. Put differently, there are \(\tilde{n}_S\) small jobs that
the oracle believes to be large, and \(\tilde{n}_L\) large jobs that the oracle believes to be
small.

\section{Related Work}

\section{Sum of Completion Times, No Release Times}

\subsection{Single Machine \(1||\sum C_j\)}

It is well-known that in the clairvoyant case, the optimal solution consists in scheduling small
jobs first, i.e., the optimal objective value is
\[
    \sum C_j^*=\frac{1}{2}n_S(n_S+1)+n_Sn_L+\frac{1}{2}n_L(n_L+1)L.
\]

In the non-clairvoyant case (NC), we clearly cannot apply the same strategy. By taking a random
permutation (uniformly), we can show that the expected value of the objective is
\[
    E(\sum C_j^{\mathrm{NC}})=\frac{1}{2}(n_S+n_L+1)(n_S+n_LL).
\]

Now let us consider the use of our semi-clairvoyant oracle (SC). If we try to apply the optimal
clairvoyant strategy, we can easily derive the objective value of a worst-case schedule, which
consists in scheduling the \(\tilde{n}_L\) large jobs that the oracle believes to be small first,
then the \(n_S-\tilde{n}_S\) small jobs, then the \(n_L-\tilde{n}_L\) large jobs, and finally the
\(\tilde{n}_S\) small jobs that the oracle believes to be large. Then, we have
\begin{align*}
    \sum C_j^{\mathrm{SC}}
        &\le\frac{1}{2}\tilde{n}_L(\tilde{n}_L+1)L\\
        &\qquad+(n_S-\tilde{n}_S)\tilde{n}_LL+\frac{1}{2}(n_S-\tilde{n}_S)(n_S-\tilde{n}_S+1)\\
        &\qquad+(n_L-\tilde{n}_L)(n_S-\tilde{n}_S+\tilde{n}_LL)+\frac{1}{2}(n_L-\tilde{n}_L)(n_L-\tilde{n}_L+1)L\\
        &\qquad+\tilde{n}_S(n_S-\tilde{n}_S+n_LL)+\frac{1}{2}\tilde{n}_S(\tilde{n}_S+1).
\end{align*}

We could try to avoid the worst-case by shuffling independently the \(n_S-\tilde{n}_S+\tilde{n}_L\)
small jobs (as seen by the oracle) and the \(\tilde{n}_S+n_L-\tilde{n}_L\) large jobs (as seen by
the oracle). This way, we get
\begin{align*}
    E(\sum C_j^{\mathrm{SC}})
        &=\frac{1}{2}(n_S-\tilde{n}_S+\tilde{n}_L+1)(n_S-\tilde{n}_S+\tilde{n}_LL)\\
        &\qquad+(\tilde{n}_S+n_L-\tilde{n}_L)(n_S-\tilde{n}_S+\tilde{n}_LL)\\
        &\qquad+\frac{1}{2}(\tilde{n}_S+n_L-\tilde{n}_L+1)(\tilde{n}_S+n_LL-\tilde{n}_LL)
\end{align*}

\subsection{Identical Machines \(P||\sum C_j\)}

\ad{TODO}

\section{Sum of Completion Times, Release Times}

\ad{TODO}

\section{Sum of Flow Times, Release Times}

\ad{TODO}

\end{document}
